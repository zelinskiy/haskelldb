\documentclass[12pt]{beamer}
\usetheme{Luebeck}
\usepackage[utf8]{inputenc}
\usepackage{listings}
\usepackage[english,russian]{babel}
\usepackage{amsmath}
\usepackage{amsfonts}
\usepackage{amssymb}
\usepackage{graphicx}
\author{Nikita Yurchenko}
\title{HaskellDB}
\setbeamercovered{transparent} 
%\setbeamertemplate{navigation symbols}{} 
\logo{} 
\institute{National University of Radioelectronics} 
\date{30.12.2016} 
%\subject{} 

\lstset{
basicstyle=\ttfamily, 
columns=fullflexible, 
numbers=left,
commentstyle=\rm
}

\begin{document}

\begin{frame}{Содержание}
\titlepage
\end{frame}

\begin{frame}
\tableofcontents
\end{frame}

\section{Обзор}
\subsection{Вступление}
\begin{frame}{Обзор}
\begin{itemize}
\item HaskellDB - Библиотека комбинаторов для языка Haskell, которые реализуют операторы реляционной алгебры типобезопасным и декларативным способом.
\item Попытка DSL (Domain Specific Language) на базе TH (Template Haskell).
\item Можно работать с БД не зная SQL, а только реляционную алгебру и Haskell.
\item Изначально написанная Эриком Мейером библиотека использовала бекенд ADO и была игрушкой теоретиков (впрочем, ей она и осталась).
\end{itemize}
\end{frame}

\subsection{Плюсы и минусы}
\begin{frame}{Плюсы и минусы}
С изменяемыми данными (в т.ч. БД) в Haskell из-за чистоты языка все традиционно обстоит довольно плохо.
\begin{columns}
\column{0.5\textwidth}
\begin{itemize}
\item[+] Интересно и необычно
\item[+] Вкусный Haskell
\item[+] Вкусная реляционная алгебра
\item[+] Ошибки сводятся на нет благодаря типам
\item[+] MySQL, PostgreSQL, SQLite
\end{itemize}
\column{0.5\textwidth}
\begin{itemize}
\item[-] Никому не нужно
\item[-] Плохо переваривает особенности конкретных СУБД
\item[-] Тормозит
\item[-] Проблемы с ленью языка
\end{itemize}
\end{columns}
\end{frame}


\section{Функции и Типы}
\subsection{Функции реляционной алгебры}
\begin{frame}[fragile]
\frametitle{Функции реляционной алгебры}
\begin{lstlisting}
restrict $ emps!E.xid .>. constant 2
project $ E.name << emps!E.name
intersect query1 query2
union query1 query2
minus query1 query2
divide query1 query2
\end{lstlisting}
\end{frame}

\subsection{Агрегатные и вспомогательные функции}
\begin{frame}[fragile]
\frametitle{Агрегатные и вспомогательные функции}
\begin{lstlisting}
top 10
unique
project $ E.xid << variance (emps!E.xid)
project $ E.xid << count (emps!E.xid)
project $ E.xid << avg (emps!E.xid)
project $ E.xid << _min (emps!E.xid)
restrict $ emps!E.xid .==. (cast "INT" (constant "2"))
order [(desc emps E.xid), (asc emps E.name)]
\end{lstlisting}
\end{frame}

\subsection{Типы}
\begin{frame}[fragile]
\frametitle{Типы}
\begin{description}
\item[Rel] Типизированное отношение
\item[Attr] Типизированный атрибут
\item[Query] Тип запроса (создается в монаде функциями)
\item[Table] Самый непосредственно относящийся к самой БД тип
\item[Expr] Тип выражения, инкапсулирует PrimExpr
\end{description}
\end{frame}

\section{Примеры}
\subsection{Описание схемы}
\begin{frame}[fragile]
\frametitle{Подключение к БД (описание БД)}
Предположим, имеется некоторая SQLite БД с таблицей emp(id, name). Создадим файл createLayout.hs, скомпилируем и запустим его
\begin{lstlisting}
module Main where
import ...

dbdescr = DBInfo "Base" (DBOptions False mkIdentPreserving) [
    TInfo "emp" [
        CInfo "id" (IntT, False)
      , CInfo "name" (StringT, False)]
  ]

main = dbInfoToModuleFiles "." "Base" dbdescr
\end{lstlisting}
\end{frame}

\begin{frame}[fragile]
\frametitle{[2] Подключение к БД (простой запрос)}
\subsection{Запрашиваем данные}
\begin{lstlisting}
module Main where
import ...
import qualified Base.Emp as E

getAllEmps db = query db $ do
  emps <- table E.emp
  project $ copyAll emps
  
withDb = sqliteConnect "DB.db"

main = do
  res <- withDb getAllEmps
  print $ res
\end{lstlisting}
\end{frame}

\begin{frame}[fragile]
\frametitle{Изменение данных}
\subsection{Меняем данные}
Расширим наш Main.hs, добавим новые функции
\begin{lstlisting}
addEmp :: Int -> String -> Database -> IO ()
addEmp id name db = insert db E.emp $
    E.xid <<- id
  # E.name <<- name

updEmp :: Int -> String -> Database -> IO ()
updEmp id name db = do
  update db
         E.emp
         (\e -> e!E.xid .==. constant id)
         (\e -> E.name << constant name)
\end{lstlisting}
\end{frame}

\begin{frame}[fragile]
\frametitle{Ссылки}
\begin{itemize}
\item \url{github.com/zelinskiy/haskelldb}
\item \url{hackage.haskell.org/package/haskelldb}
\item \url{hrisdone.com/posts/haskelldb-tutorial}
\item \url{https://github.com/MasseR/haskelldb-example}
\item \url{citeseerx.ist.psu.edu/viewdoc/
download?doi=10.1.1.136.3828&rep=rep1&type=pdf}

\end{itemize}
\end{frame}

\end{document}